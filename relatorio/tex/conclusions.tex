\section{Conclusões}
	Pode-se concluir que:
	\begin{itemize}
		\item O valor de $K$ é muito importante para a resolução do problema como pode ser visto nos tempos dos experimentos 1 e 2. Um maior valor de $K$ adiciona complexidade ao problema.
		\item O intervalo em que Max 3-SAT tem comportamento exponencial é quando $M \geq 430$ e $M \leq 600$ para $N = 100$, ou seja com $M / N \geq 4.3$ e $M / N \leq 6.0$.
		\item O intervalo em que Max 2-SAT tem comportamento exponencial é quando $M \geq 200$ e $M \leq 300$ para $N = 100$, ou seja com $M / N \geq 2.0$ e $M / N \leq 3.0$.
		\item Enquanto o valor de $M/N$ é mais grande, é mais provável que o percentagem de instâncias satisfazíveis seja menor.
	\end{itemize}
	
\clearpage