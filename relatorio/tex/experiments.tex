\section{Experimentos e resultados} 
 
\subsection{Experimento 1} 
	\subsubsection{Descrição}
		Para $K = 3$ (ou seja, 3-SAT) e $N = 100$, levantar a curva de resposta de tempo, e apresentá-la sobreposta à curva de percentagem de problemas satisfazíveis. Cada ponto deve ser obtido a partir de pelo menos 100 instâncias geradas aleatoriamente; preferivelmente, utilizar 1.000 instâncias. Apresentar e discutir o formato do gráfico.
	\subsubsection{Resultados}
	
 \subsection{Experimento 2} 
	\subsubsection{Descrição}
		Para $K = 2$ (ou seja, 2-SAT) e $N = 100$, levantar a curva de resposta de tempo, e apresentá-la sobreposta à curva de percentagem de problemas satisfazíveis. Cada ponto deve ser obtido a partir de pelo menos 100 instâncias geradas aleatoriamente; preferivelmente, utilizar 1.000 instâncias. Apresentar e discutir o formato do gráfico.
	\subsubsection{Resultados}

\subsection{Experimento 3} 
	\subsubsection{Descrição}
		Apresentar 5 gráficos mostrando o tempo de execução em função de $N$ para $K = 3$. Em cada gráfico, o valor de $M / N$ deve ser fixo. Os cinco gráficos devem ser feitos para $N$ variando de 100 a 1000, em intervalos de 100. Os valores de $M / N$ de cada um dos 5 gráficos são 1; 3; 4,3; 6; e 8. Discutir a natureza da curva obtida em cada caso, se polinomial ou exponencial.
	\subsubsection{Resultados}

\subsection{Experimento 4} 
	\subsubsection{Descrição}
		Apresentar 5 gráficos mostrando o tempo de execução em função de $N$ para $K = 2$. Em cada gráfico, o valor de $M / N$ deve ser fixo. Os cinco gráficos devem ser feitos para $N$ variando de 100 a 1000, em intervalos de 100. Os valores de $M / N$ de cada um dos 5 gráficos são 1; 3; 4,3; 6; e 8. Discutir a natureza da curva obtida em cada caso, se polinomial ou exponencial.
	
	\subsubsection{Resultados}

\clearpage 
